\documentclass{article}
\usepackage{amsmath}
\usepackage{epsfig}
\begin{document}
\centerline{Solution of Kepler's equation, $ 0 < e < 1$}

\vskip .2 in
Let $f(x) = x - e\sin(x)$, i.e, Kepler's equation becomes $f(x) = M$. Since
$M$ is an angle we may assume $M \in [0,2\pi).$ Moreover, we may assume
$M \in [0,\pi].$ Else, if $x$ satisfies $f(x) = 2\pi - M$, then
using $\sin(x) = -\sin(2\pi -x)$,
$$
f(2\pi -x) = 2\pi - x - e\sin(2\pi -x) = 2\pi - f(x) = M.
$$
Since $f^{\prime}(x) = 1 - e\cos(x) > 0$, $f(x)$ is strictly increasing,
continuous, $f(0) = 0, f(\pi) = \pi$, so a unique solution exists by the
Intermediate Value Theorem.

\vskip .1 in
% first diagram here
\epsfig{file=fig1.eps,height=4in,width=2in,angle=270}

\vskip .2 in
Method 1: Simple iteration.
\vskip .1 in
Let $E_0 = M, \dots, E_n = g(E_{n-1}),$ where $g(x) = M + e\sin(x).$ Then
$$
|g^{\prime}(x)| = |e\cos(x)| \le e < 1,
$$
so $g$ is a contraction and has a unique fixed point $E$. Moreover, the
$E_n$ converge to $E$, which is the solution of Kepler's equation.
\vskip .2 in
Method 2: Newton's method.
We may apply Newton's method to find a root of $f(x) - M  = 0.$ This
gives $E_0 = M$,
$$
E_n = E_{n-1} + \frac{M - f(E_{n-1})}{f^{\prime}(E_{n-1})}
=  E_{n-1} + \frac{M + e\sin(E_{n-1}) - E_{n-1}}{1 - e\cos(E_{n-1})}.
$$
\vskip .2 in
Method 3: Binary search
\vskip .1 in
$M - f(x)$ is strictly decreasing with a unique zero. Define a sequence
$I_n$ of closed intervals whose intersection contains the root as follows:
$I_0 = [0,\pi]\dots I_n = $ the right half of $I_{n-1}$ if $f(\text{midpoint})
> 0.$ Otherwise, take $I_n$ to be the left half of $I_{n-1}.$
\vskip .1 in
\epsfig{file=fig2.eps,height=4in,width=2in,angle=270}
\vskip .2 in
Method 4: Series expansion in powers of $e$.
\vskip .1 in
The series
$$
 E = M + \sum_{n=1}^{\infty} A_n e^n,
$$
where
$$
A_n = \frac1{2^{n-1}n!} \sum_{k=0}^{[\frac{n}2]} (-1)^k \binom nk
(n-2k)^{n-1} \sin((n-2k)M),
$$
converges for $ e < .662743\dots.$
\vskip .2 in
Method 5: Fourier expansion
\vskip .1 in
$E - M = e\sin(E)$ is an odd, continuous, $2\pi$ periodic function of $M$,
hence can be expanded in a Fourier sine series:
$$
	E = M + 2\sum_{n=1}^{\infty} A_n \sin(nM),
$$
where 
$$
A_n = \frac1{\pi} \int_0^{\pi} e\sin(E)\sin(n\theta)\,d\theta.
$$
Integrating by parts, we have for $A_n$
$$
\frac1{n\pi}\int_0^{\pi}\cos(n\theta)d(e\sin(E)) 
=
\frac1{n\pi}\int_0^{\pi}\cos(n\theta)d(E - \theta).
$$
The $d\theta$ term vanishes, and making the the substitution
$\theta = E - e\sin(E),$ we have
$$
\frac1{n\pi}\int_0^{\pi}\cos(nE - ne\sin(E))d(E) = \frac{J_n(ne)}{n},
$$
where
$$
J_n(x) = \frac1{\pi}\int_0^{\pi}\cos(n\theta - x\sin(\theta))d\theta 
$$
is the {\it Bessel function of the first kind.} In summary, we have
the expansion
$$
E = M + 2\sum_{n=1}^{\infty} J_n(ne)\frac{\sin(nM)}{n}.
$$
\end{document} 
\bye
