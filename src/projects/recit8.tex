\documentstyle{article}
\begin{document}
\sf
\begin{center}
{\bf Recitation 7 Activities  Mathematics 397 002/Honors }
\end{center}
\bigskip
The following  problems are to be worked in groups. The groups for
this recitation are: 
\begin{itemize}
\item Group I: E. Miller, Willey, Davis
\item Group II: Hobbs, Champagne, Newell
\item Group III: S. Miller, Chawan, Taylor
\end{itemize}

Work on scrap paper together, and then write up the final solution on this
sheet and hand in. (You can use another sheet of paper if there is not
room on this one.)
\par
\bigskip

Problem 1: The purpose of this exercise is to do the reaction time measurement
again, but this time to include an error analysis. I.e, this time your answer
should have the form: Our reaction time estimate is $X$, with a possible error
of $\pm Y\%.$ Let us agree this time upon a common method of measurement. 
Hold a ruler just above the gap between your partner's thumb and forefinger and
drop it at a randomly chosen time. Read off the distance it fell from the scale
on the ruler. Your main job is to figure out some rational way of assessing the
accuracy of this measure of distance, and then to use the methods discussed
in class to assess the accuracy of the resulting reaction time.    
\vskip 2 in
\par
Problem 2: { \bf An identity from Thermodynamics. } Consider three variables
$x,$ $y,$ and $z,$ which are related to one another by an equation of
the form $F(x,y,z) = 0,$ where the function $F$ has continuous and nonzero
first partial derivatives with respect to each variable. Such an equation
is called an {\bf equation of state.} Consider a particular point $(x_0,
y_0, z_0)$ in the domain of $F$. Suppose we hold $ z = z_0.$ Then the
equation of state can be solved for x as a function of y. We denote the
derivative of this function as $ (\frac{\partial x}{\partial y})_z,$ and
read this as ``the derivative of x with respect to y at constant z.'' Similarly,
if any other variable is held constant, we can express either of the 
remaining variables as functions of the third by solving the equation of
state, and then calculate the derivative of this function. Show that
the following identity holds:
$$
(\frac{\partial x}{\partial y})_z(\frac{\partial y}{\partial z})_x(\frac{\partial z}{\partial x})_y  = -1.
$$
\vskip .1 in
Suggestion: differentiate both sides of the equation of state with respect
to x at constant z using the chain rule, i.e., differentiate the function
$g(x) = F(x,y(x),z),$ where $ y(x)$ is the function that results from solving
the equation of state for y as a function of x. Repeat for each of the 
variables.  

\end{document}
