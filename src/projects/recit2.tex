\documentstyle{article}
\begin{document}
\sf
\begin{center}
{\bf Recitation 2 Activities  Mathematics 397 002/Honors }
\end{center}
\bigskip
The following  problem is to be worked in groups. The groups for
this recitation are: 
\begin{itemize}
\item Group I: Newell, Chawan, E. Miller
\item Group II: Hobbs, Willey, Champagne
\item Group III: Davis, S. Miller, Taylor
\end{itemize}

Work on scrap paper together, and then write up the final solution on this
sheet and hand in.
\par
\bigskip
Problem:  Consider a regular tetrahedron. This is the pyramid-shaped solid
whose 4 faces are congruent equilateral triangles. We define the {\bf centroid
} of the tetrahedron as the point whose coordinates are the average of the
coordinates of its 4 vertices. Each triangular face has a centroid which
is the average of the 3 vertices of that face. Now imagine a line drawn 
from the centroid of the tetrahedron to the centroid of each face. There
are 4 such lines. Calculate the angle between any pair of them.
\par
\smallskip
This angle arises in Physical Chemistry: it is the bond angle in a molecule
of methane.
\par
\smallskip
Suggestion: you can make life a lot easier by placing the vertices of your
tetrahedron at well-chosen points. I suggest putting one vertex on the 
positive z-axis and the remaining 3 in a plane parallel to and below the
xy plane, chosen so that  the origin is the centroid. Or, you might try
a configuration in which all but one of the vertices lie on a coordinate
axis.
\end{document}
