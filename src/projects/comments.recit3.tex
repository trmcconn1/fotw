{\bf Comments on recitation 3 work.}
\vskip 1 in
This was quite a difficult pair of problems and you all did pretty well
with them. In problem 1 you should treat the case where 
$|v\cdot w| = |v||w|$ as a special case. In general, one has
$|v\cdot w| \le |v||w|$, a result known as {\bf Schwarz's inequality.}
It is rather obvious from the geometric definition of the dot product, since
it amounts to saying that $|\cos \theta| \le 1$, but is not at all obvious
from the component expression for the dot product,
$$
| a_1b_1 + a_2b_2 + a_3b_3| \le \sqrt{a_1^2 + a_2^2 + a_3^2}\sqrt{b_1^2 + 
b_2^2 + b_3^2},
$$
confirming once again how useful it is to retain both the geometric and
algebraic points of view.  Equality holds in Schwarz's inequality exactly
when v and w are parallel. In that case you could take $\beta = 0$ and
$$
\alpha = \frac{v\cdot u}{|v|^2},
$$ 
but there are infinitely many other choices that will work as well.

Nobody got the general case of Lie's identity. Here is how it works. Let
us choose $\alpha$ and $\beta$ as in Problem 1. Then, expanding and
collecting terms,
$$
\multline
(u\times v)\times w + (w \times u)\times v + (v \times w) \times u = 
(z\times v)\times w + (w \times z)\times v + (v \times w) \times z  \\ + 
\alpha ((v\times v)\times w + (w \times v)\times v + (v \times w) \times v) \\ + 
\beta ((w\times v)\times w + (w \times w)\times v + (v \times w) \times w ). 
\endmultline
$$

The terms preceded by $\alpha$ and $\beta$ reduce to zero by case 1, so the
net effect has been to replace the original u by a vector z which is
perpendicular to both v and w. Now repeat this argument with v playing the role
of u, and again with w playing the role of u. The end result is to replace
the left side of Lie's identity with the same expression having all three
vectors perpendicular. This is zero by case 2. 

The study of products satisfying Lie's identity forms a branch of modern
mathematics known as Lie Algebras. Other Lie products arise in Quantum 
Mechanics.  
 

\bye
