{\bf Comments on recitation 6 work.}
\vskip 1 in

The calculations needed to find the actual distance travelled by the golf
ball turned out to be rather herculean, almost as herculean as the effort
needed to drive a golf ball 300 yards. Most of you fairly quickly found the
velocity vector for the ball at time t,
$$
r(t) = 198.9ti + (72.4-32t)^2j
$$

which  comes from resolving the initial velocity vector into components and
using the fact that thereafter the vertical component decreases linearly
at rate 32 feet per second per second. The
initial velocity vector can found using the range formula, equation 15 on
page 766. The time until impact is double the time until maximum height, 
which occurs when the j component is 0: $t = 72.4/32 = 2.26.$ Using the
arc-length formula the problem reduces to calculating
$$
\int_0^{4.5}\sqrt{(198.9)^2 + (72.4-32t)^2} \, dt = 914.23 \text{ft.}
$$

Your estimates of reaction time fell in the range .15 - .3 seconds, which
is quite reasonable. At last year's olympic 100m dash, Linford Christie of
Great Britain was disqualified because sensors in his starting blocks
detected that his reaction time to the gun was faster than .1 sec. It is
generally felt that no human can possibly react that fast, although this
is still somewhat controversial. A dollar bill is exactly 6 inches
long ( a useful fact to know if you're ever stranded without a ruler,) which
makes catching a falling dollar right at the limit of acheivable reaction
time.

\bye
