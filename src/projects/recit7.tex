\documentstyle{article}
\begin{document}
\sf
\begin{center}
{\bf Recitation 7 Activities  Mathematics 397 002/Honors }
\end{center}
\bigskip
The following  problems are to be worked in groups. The groups for
this recitation are: 
\begin{itemize}
\item Group I: Hobbs, Willey, Taylor
\item Group II: Chawan, Davis, Newell
\item Group III: S. Miller, E. Miller, Champagne
\end{itemize}

Work on scrap paper together, and then write up the final solution on this
sheet and hand in. (You can use another sheet of paper if there is not
room on this one.)
\par
\bigskip

Problem 1: Attached is a photocopy of a portion of a USGS topographical map
showing the area between Tully, N.Y., and Fabius, N.Y. (Approximately 20 miles
SSE of the University.) This region is part of an interesting geological
formation known as the ``Valley Heads Moraine.'' It is a nearly continuous
ridge of material left behind by the glacier and extending across much of
upstate New York. To the west, the moraine forms a natural dam at
the end of a series of very deep valleys which have filled with water -- these 
are the "Finger Lakes." In our region, the moraine forms the divide
between two major watersheds: rain falling to the north of the divide flows
into the St. Lawrence river and, ultimately, the North Atlantic, whereas rain
falling to the south  flows into the Susquehanna river and,
eventually, Chessapeaque bay.
	
In this exercise you are to trace the curve on the map which marks the location
of the divide. (To simplify matters, I have marked a starting point and
the approximate starting direction for the divide.)

Suggestion: It is enough to mark the boundary of either the St. Lawrence or
Susquehanna watersheds. To find the boundary of a watershed, choose an arbitrary
point 
as starting point
and draw a curve which heads directly ``uphill''. As we shall see
later, such a curve must be perpendicular to each contour line it crosses.
Continue this curve until it is no longer possible to keep going uphill, and
mark the point at which it would begin to descend again. Do this for many
different starting points. The points so obtained will mark out the boundary
of the watershed. (Note that river courses are examples of such ``uphill''
curves.) On the map, Fabius brook belongs to the Susquehanna watershed, while
Butternut creek belongs to the St. Lawrence watershed.
\vskip 2 in
\par
Problem 2: Consider the function shown below
$$
f(x,y) = xy\frac{x^2-y^2}{x^2 + y^2}, (x,y) \ne (0,0).
$$
Also define $f(0,0) = 0.$ Show that $f_{xy} \ne f_{yx}.$ 

Suggestion: first calculate $f_y(x,0) $ by considering separately the cases
$x = 0 $ and $ x\ne 0.$ Do the same for $f_x(0,y).$ Then differentiate these
functions.
\end{document}
