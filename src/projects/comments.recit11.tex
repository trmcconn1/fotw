\magnification = \magstep 1 
{\bf Comments on recitation 11 work.}

\vskip .1 in
Problem 1. The volumes of the unit spheres in all dimensions are known
and are given by the following formula, where $V_n$ denotes the volume of
the unit sphere in n-dimensional space:
$$
V_{2k} = \frac{\pi^k}{k!},
$$
and
$$
V_{2k+1} = \frac{2^{k+1}\pi^k}{(2k+1)!!},
$$
where the {\it double factorial} is defined by $ n!! = n(n-2)(n-4)...1 $
for odd n. Thus, starting with the unit circle in 2-dimensions, the first
six such volumes are $\pi, \frac{4\pi}3, \frac{\pi^2}2, \frac{8\pi^2}{15},
\frac{\pi^3}6, \frac{16\pi^3}{105}.$ For further information, see
W.H. Fleming, {\it Functions of Several Variables,} Addison Wesley, Reading
(1965), 182-183.
\vskip .1 in 
Problem 2. All 3 groups gave the answer 1, which is clearly wrong if you
think about it. A tetrahedron with side length 1 will fit inside a unit
cube with lots of empty space left over. The correct answer is
$\frac{\sqrt{2}}{12}.$ To see this, first use the tetrahedron with
vertices (1,0,0), (0,1,0), (0,0,1), (1,1,1), which has all side lengths
equal to $\sqrt{2}.$ The area of the base ( in the plane of the first 3
vertices ) is seen to be $\frac{\sqrt{3}}{2}$ by simple trigonometry. The
altitude is the distance from (1,1,1) to the centroid of the base,
$(\frac13, \frac13, \frac13).$ This equals $\frac2{\sqrt{3}}.$ Since the
regular tetrahedron is, in particular, a right pyramid, the volume is
given by $\frac13Ah.$ Thus the volume equals $\frac13.$. Finally, multiplying
the side lengths by $\frac1{\sqrt{2}}$ multiplies the volume by the cube of
this number.

\bye
