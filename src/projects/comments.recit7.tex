{\bf Comments on recitation 6 work.}
\vskip 1 in

All 3 groups pretty much traced the divide correctly from the ``X'' towards
the Northeast. It climbs up Matson Hill, runs along the crest, descends
and crosses a col between Matson Hill and Chase Hill, and then climbs the
latter hill, crossing Chase Rd just East of the intersection with
Swift Rd. (By the way, I live on Swift road about a half mile North of the
top of the map. The divide goes down the middle of the road at the end
of my driveway.) Tracing the divide towards the West from the ``X'' is
much more difficult. I think Kyle, Mark and Shawn have it basically right:
it loops back towards the SE, climbing to the summit of Fellows Hill, then
turns towards the West and climbs to the summit of Jones Hill, before
descending again to pass the ``X'' location about a mile to the SW.
The small pond to the SE of Apulia Station is in the St. Lawrence watershed.

The counterexample involving mixed partial derivatives is taken from
the book {\it Counterexamples in Analysis,} B.R. Gelbaum and J.M.H. Olmstead,
Holden Day, San Francisco, 1964. It is example 10 on p. 120. All 3 groups
were a bit careless about division by zero. The calculation of both
$f_x(0,0),$ and $f_y(0,0),$ is simple, but must be handled separately:
$$
f_x(0,0) = lim_{h\to 0}\frac{f(h,0) - f(0,0)}h = \frac{0-0}h = 0.
$$

\bye
