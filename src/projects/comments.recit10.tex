{\bf Comments on recitation 10 work.}
\vskip 1 in

I lined all of your maps up and the centroids found by all 3 groups
essentially coincided. Somewhere just SW of Topeka Kansas ( in a cornfield
in all probability ) the sacred point lies. Perhaps it is marked by a
plaque.
\vskip .1 in
I am inclined to trust you on this one.
\vskip .1 in

The actual coordinates given differed quite a bit because of different choices
of origin. Indeed, the one complaint I have about your work is that nobody
made a clear note of the location of the origin. I did manage to deduce
the location, however. Here are the results:
Group 1: Point at (13.9,7.8), origin located just SW of San Diego on the
coastline. Group 2: Point at (18.5, 14.9), x axis tangent to the southern
tip of Florida, y axis tangent to the  coast of California. Group 3: Point
at (13.9, 17.4), origin located above the
NW corner of the US so that the NW corner of the US is at (3,1), y-axis
running E-W (somebody had to be different.)   

\vskip .1 in 
Apparently running short on time, group 1 used an approximation
method to estimate the y-coordinate. (Well, the whole thing is an
approximation anyway, so I guess this amounts to an approximation to an
approximation.) As far as I can tell, they broke the range of y values
into 5 blocks and used a representative count of horizontal squares for
each block. Points for creativity, but the explanation was a bit fuzzy so
we'll call it a wash. 


\bye
