\documentstyle{article}
\begin{document}
\sf
\begin{center}
{\bf Recitation 1 Activities  Mathematics 397 002/Honors }
\end{center}
\bigskip
The following two problems are to be worked in groups. The groups for
this recitation are: 
\begin{itemize}
\item Group I: Champagne, Chawan, Davis
\item Group II: Hobbs, E. Miller, S. Miller
\item Group III: Taylor, Willey, Newell
\end{itemize}

Work on scrap paper together, and then write up the final solution on this
sheet and hand in.
\par
\bigskip
Problem 1. Using only the parallelogram law definition of vector addition,
demonstrate the associative law of vector addition for vectors in the plane.
Your answer should be in the form of a diagram(s), clearly labeled, with
perhaps a paragraph of explanation if necessary. 
\vskip 2 in
Problem 2. Consider 3 nonzero vectors $u,v,w,$ in space which do not lie in
a common plane. Assume the initial points of all three vectors are placed
at the origin. Describe, in geometric terms, the set of all terminal points
of the vectors $\alpha u + \beta v + \gamma w$ obtained as the scalers
$\alpha,\beta,$ and $\gamma$ range over all nonnegative numbers such that
$\alpha + \beta + \gamma = 1.$
\vskip .2 cm
Suggestions: Study the special case $u = i, v = j, w = k$ and plot the
points obtained for various choices of $(\alpha, \beta, \gamma)$ :
$(1,0,0), (0,1,0), (0,0,1), (\frac13,\frac13,\frac13), (\frac12,\frac12,0),
\dots .$ Formulate and study an analogous question for plane vectors. 

\end{document}
