{\bf Comments on recitation 8 work.}
\vskip 1 in

Hoo boy... I feel a tirade coming on.
\vskip .1 in
Problem 1. Estimates of accuracy ranged from 40\% to 100\% which seems 
reasonable, but the work was rather sloppily done. I seriously doubt whether
any of you could look at your own work a year from now and have any idea
what the numbers mean. When you present data, always state clearly exactly
what was measured and what units of measurement were used. I managed to
figure out (I think) that in the expression $\frac{\Delta S}{S}$ the
${\Delta S}$ represents the largest minus the smallest measure of distance
and that the $S$ in the denominator represents the smallest measured
distance. Nobody tried to explain why these choices were made. 

\vskip .1 in
One difficulty with this problem is that the quantity being measured, human
reaction time, is actually a random variable with a certain probability
distribution, rather than a deterministic quantity. Accordingly, a proper
error analyis really belongs to the subject of {\it Statistics.} On
statistical grounds it could be argued that a better estimate of error
is given by the standard deviation of the measurements, but the largest
minus smallest does provide a (conservative) upper bound to the standard
deviation, and one which is easily computed. 

\vskip .1 in
Concerning significant digits: If I measure the length of my thumb with a
dime-store ruler and report the answer as 3.078321 inches, you should laugh
at me because there is no way my answer could be that accurate. When a
scientist reports results of a measurement to the outside world, the
number of significant digits is intended to convey information about the
accuracy of the measurement. (The number of significant digits is one plus
the number of digits dispayed to the right of the decimal when a number is
expressed in scientific notation: 3.07 and 3.00 have 3 significant digits,
.0003 has one. ) It is reasonable to retain several additional digits during
calculations so that round-off errors do not accumulate, but the final
answer should always be rounded to an appropriate number of significant
digits. Question: How do I know how many significant digits are appropriate
in a result ? Answer: I do an error analysis like the one you did. If there
is a 10\% or more relative error, I have no right to claim more than one
significant digit.

\vskip .1 in
It is curious that you all seemed to understand the
relationship
$$
	\frac{dT}{T} = \frac12\frac{dS}{S} 
$$
and yet many of you couldn't do problem 3 on the test. 
\vskip 1 in
Problem 2. Students often remember the simple one variable chain rule
$$
\frac{df}{dx} = \frac{df}{dy}\frac{dy}{dx},
$$
where $y$ is a function of $x$, by imagining that the factor $dy$ on the
right-hand side can be ``cancelled''. Indeed, this mnemonic quality is often
touted as one of the strengths of the Leibniz system of notation. 
Unfortunately, the Leibniz system can also be misleading, as this example
shows. (Cancellation of the $\partial x$ symbols leads to the incorrect
value 1 rather than -1.) All of you wrote down the equation for the
differential
$$
dF = \frac{\partial F}{\partial x}dx +  \frac{\partial F}{\partial z}dz = 0
$$
and then ``solved'' for the ratio $\frac{\partial z}{\partial x},$ thus falling
into precisely the kind of trap this example is intended to warn against. 
It is important to realize that in the above equation the symbols $dx$ and
$dz$ represent arbitrary changes in the independent variables: their
ratio has no particular significance. The proper solution is to
write the following, in which z represents the (implicitly defined) function
of x and y that results from solving the equation of state:
$$ 
 \frac{\partial F}{\partial x} +  \frac{\partial F}{\partial z}\frac{\partial z}{\partial x} = 0.
$$

\bye
