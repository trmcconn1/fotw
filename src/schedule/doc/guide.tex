\documentstyle{article}
\begin{document}

\title{Mathematics Department Schedule Program User's Guide}
\author{Terry McConnell}
\date{\today}
\maketitle
\tableofcontents
\sf

\section{Requirements}
Schedule requires 2 Megabytes of hard drive space and 470 Kbytes of 
conventional memory in order to run. It runs under DOS 6.22 or higher.
(It will probably even run under DOS 3.0, but who cares?) It will also
run in a DOS box under MS windows 3.1 and 95. In fact, this is the
preferred way to run the program. Since schedule has no built-in editor,
running it in a multi-tasking environment allows the user to run the
editor of her choice, either under Windows directly, or in another DOS
session. The program can also be compiled for Unix, but only a command line
interface is currently available there. The Unix version has been 
thorougly tested under NextStep 3.2 and Linux 2.0.6. The DOS version 
can be built from the source code provided using Borland C/C++ 4.5. It will
probably build under any reasonable compiler, but the Makefile may need
to be edited.
In some of the discussion below we assume that the root directory of drive
C: is on the command search path. This is not necessarily true, but usually
is.

\section{DOS Installation}
Installing the schedule program is very simple. The program is distributed
in the form of a {\it zip archive,} as created, e.g., by the shareware
program {\it pkzip.} A zip archive is a single file that contains, in
a compressed form, an entire directory structure including all subdirectories
and their files. In order to be used for anything these directories and
files need to be extracted from the archive, either using the {\it pkzip}
program itself, or with some compatible program. Assuming you have
{\it pkzip} properly installed on your computer, and  disk1
in your floppy drive a:, you would begin the
installation by by following this dialogue: \smallskip
\begin{verbatim}C:\> COPY A:SCHEDULE.ZIP C:\
C:\> PKUNZIP -d SCHEDULE.ZIP
C:\>DEL SCHEDULE.ZIP
\end{verbatim}
\smallskip
This will result in the creation of a new directory called ``schedule''
in the root directory of your C: drive. The schedule directory should
contain a number of subdirectories which in turn contain other directories
and files. To verify that all is well, do the following: \smallskip
\begin{verbatim}
C:\>cd schedule
C:\>dir

Volume in drive C is MS-DOS_62
Volume Serial Number is 3B31-07E1
Directory of C:\SCHEDULE

.           <DIR>          07-01-96     1:45p
..          <DIR>          07-01-96     1:45p
INCLUDE     <DIR>          07-01-96     1:45p
SRC         <DIR>          07-01-96     1:45p
README           2,503     06-22-96     6:25p
DATA        <DIR>          07-01-96     1:45p
DOC         <DIR>          07-01-96     1:45p
EXAMPLES    <DIR>          07-01-96     1:45p
BACKUPS     <DIR>          07-01-96     1:45p
SCRIPTS     <DIR>          07-01-96     1:45p
       10 file(s)               2,503 bytes
                        377,208,832 bytes free
\end{verbatim}
\smallskip 
If you see this listing, or something similar, then everything is OK so far.

If you do not have or cannot install {\it pkzip} or a compatible program, or if the above fails for some reason, you should be able to copy all of the
necessary files from disk2. Create the schedule 
directory yourself using the MKDIR command. Then use, e.g., the XCOPY
command to copy the contents of each of the above directories from 
disk2. See your DOS documentation for details on the MKDIR
and XCOPY commands.

The only thing that remains to be done is to create the program's configuration
file. Schedule refers to a file called the {\it configuration file} when it first
starts up to learn various pieces of information it needs in order to run
properly. The easiest way to create this file is just to copy the file
called rcfile in the examples directory to the root directory of your
C drive:
\smallskip
\begin{verbatim}
C:\SCHEDULE>copy examples\rcfile \schedule.rc
\end{verbatim}
\smallskip
You don't really have to put schedule.rc in the root directory -- it can
be placed in any directory on the command search path. (Consult your
DOS documentation to learn about the command search path and how to
change it if necessary. If you don't mind leaving schedule.rc in the
root directory then don't worry about the search path.) 
Use an editor to examine the contents of schedule.rc. You will notice that it
contains many lines beginning with the \# character, and other lines beginning
with the word ``set''. Lines beginning with \# are completely ignored by the
program (they are called {\it comments.}) They are put in the file to explain
what the various entries mean. You can add additional ones if you wish. 
The other lines set the values of  internal variables the program
uses. For example, the line
\smallskip
\begin{verbatim}
set CWD  C:\schedule
\end{verbatim}
\smallskip
informs the program that it should begin its operations by changing to the
directory C:$\backslash${schedule} (its {\bf C}urrent {\bf W}orking {\bf D}irectory )
before doing anything else. You would need to change this line only if
you had changed the name of the schedule directory or moved it somewhere other
than the root directory ( you didn't, did you ?) In fact, you can probably leave
this file untouched except for the line that sets the value of SEM, the
current semester. You should change the default entry, ``f97'', which 
describes the Fall 1996 semester (that's right, 96, not 97,) to whatever
the current semester is. Use "s" for a spring semester and the last two
digits of the current academic year. ( The program uses the convention that
the academic year is the year in which the Spring semester occurs -- that
explains the ``97'' above.)

That's it! The program is now installed. All you have do is type ``Schedule''
in order to run it at the DOS prompt, right? Wrong. Unfortunately, right
now the program is named C:$\backslash${SCHEDULE}$\backslash${SRC}$\backslash${SCHEDULE}$\backslash${SCHEDULE.EXE}. In order
to run it, you can either type out that mouthful each time, or else change
to the Schedule$\backslash${src}$\backslash${schedule} directory
 and then type ``schedule''. To
avoid all this typing every time you want to run the program, you could 
move the executable (schedule.exe) to any directory on the command search
path. For example, if you say
\smallskip
\begin{verbatim}
C:\SCHEDULE> copy src\schedule\schedule.exe \
\end{verbatim}
\smallskip
then all you will ever have to do to run the schedule program is type ``schedule''. Actually, it's not such a good idea to clutter up your root
directory with files. A better plan would be to create a new directory
called C:$\backslash${bin} if it doesn't already exist, move the schedule executable there,
and then make sure that bin is on the command search path. If you don't know
how to do that and don't care to learn, then get somebody who is well versed
in DOS to help you.

Finally, the schedule program (we'll just call it ``schedule'' from now
on) can be invoked with a number of command line switches that affect its
mode of operation. If you have ever given the DOS command dir /p then you
have used a command line switch: the /p is a switch that causes dir to pause after
each screenfull of information. Schedule uses the Unix convention of 
prefixing each switch with the hyphen character rather than the / character.
To see a list of available switches, give the command schedule -help. It is
unlikely that you will ever need to use any switches, but you should be
aware that they exist.

\section{Overview}
Despite its name, the schedule program does not actually create a departmental
schedule. Rather, it helps to organize and assist the process of creating
and managing a schedule. In this section we shall give an overview of the
process of creating a schedule of classes and how this program can be used
to help. Throughout, we shall consider the practices and policies of the
Syracuse University Mathematics Department as they were at the time the
program was written. However, the program is designed to be adaptable to 
almost any reasonable academic scheduling situation.

Schedule is a database management program: it reads a set of data from files
on a disk drive into the computer's memory, and then allows the user to
organize and display that data in various ways. Unlike many databases, all
the data for schedule are kept in plain ascii (i.e., human readable) format.
Also unlike many databases,  schedule  itself cannot be used to
modify the data -- it is up to user to maintain the data files using the
editor program of her choice. Schedule has no built-in editor ( although
it is possible to escape to an editor program temporarily using the edit
command of the shell.) If a word-processor is used to edit the data
files it is very important that the files be re-saved in plain ascii format.
Otherwise, most word processors will put special formatting symbols in the
file that will confuse the program. 

The data for each semester is located in a subdirectory of the data 
subdirectory. For example, the data for the fall semester of 1996 will be
found in C:$\backslash${schedule}$\backslash${data}$\backslash${f97}. There are
4 data files named {\bf people}, {\bf jobs}, {\bf rooms}, and {\bf time.}
The main building block of an academic schedule is the class, and
describing a class involves naming the person who is in charge, the job
that person does ( lecturer, recitation leader, ...), the room in which
the class meets, and the times the class meets. Each of the data files
describes the possible values each of these components of a class can take
on in a given semester. (There is one additional component of a class
that is not described here -- students -- but schedule does not attempt
to keep track of student enrollments.) All of these data files are
described in detail in section 3 below.

If these 4 data files are the only data files, then how does schedule know
which teacher goes with which class, and more importantly, how classes
are organized into courses, etc? The answer is that there is one additional
file that goes with each semester, a file  that describes the {\it structure} of
the schedule for that semester. These files are also found in the data
directory. For example, the file for the Fall 1996 semester is called
sched.f97. (This explains why we use 3 letter names for a semester: DOS only
allows 3 letters for filename extensions.) Without going into great detail
at this point, let us look at a short piece of sched.f97:
\smallskip
\begin{verbatim}
course 521 TBA
class 001 C-313 Kim tth10 Lecturer
class 002 C-313 Watkins mwf9 Lecturer
\end{verbatim}
\smallskip

These lines define a course named ``521'' which is comprised of two
sections named ``001'' and ``002''. The first word on each line is an
example of a {\it command.} For example, the class command causes an
internal data structure to be created and fills that structure with
information gotten from the remaining words on the command line. The word
immediately following the word class provides a name for the class, and
the remaining words refer to actual entries in the rooms, people, time, and
jobs files respectively. Actually, any of the
commands can be given directly to schedule while is running. In one of its
modes, the program provides a command-line interface at which the user
types a command in response to a prompt. (See the ``Actions/shell'' menu
 choice.) All of the above lines could be entered at the prompt, but it
would be extremely cumbersome to do this every time the program is run. 
Schedule allows you to collect any set of commands into a file which it will
then read and interpret exactly as if you had typed them at the prompt. (In
programming parlance, such a file of commands is called a {\it script} or
{\it batch file.})

The act of creating a schedule, then, consists of creating the 4 data files
and the schedule batch file for a given semester. Since large parts of
the these files remain the same from one semester to the next, this process
is not as tedious as it sounds. Indeed, the program automates a large part
of it. (See the description of the Actions/New Sem menu choice below. )

Let us now outline the entire process of creating a schedule, from its inception
up to the time students show up for the first day of class. The first step
is to determine {\it which } courses should be offered. In the case of
undergraduate courses the course offerings are completely determined by the
academic catalogue  and will be the same every semester. A number of
graduate courses must also be offered each semester. Beyond this, the
decision of which courses to offer is made by a committee of faculty. 

The next step is to determine how many sections of each course should be
offered in order to meet expected student demand. After this is done, the
schedule program should be used to set up a ``blank'' schedule for the
semester: All of the courses and classes are defined in the batch file, but
classrooms, teachers, and times are left as TBA. The data files should be
brought up to date as much as possible. At this point the audit feature of
schedule can be used to check whether there are sufficient personnel to
cover the budgeted classes, and  corrective action may be taken if necessary. 

A committee of faculty next determines which faculty should teach which 
classes, and the Associate chair of the department then  decides
rooms and times
for the faculty-taught courses and sections. These decisions are based upon
faculty preferences gleaned from a questionaire. They are not easily 
automated since they involve issues of fairness and a knowledge of the
recent teaching history of each faculty member. Once this step is done the
batch file and the data files should again be brought up to date. 

Next, rooms and times are assigned to all the TA-taught courses and classes.
The schedule program can be used to assist in this process by checking for
room and time conflicts and by using the roomchart feature. Finally, teaching
assistants are assigned to teach  the remaining courses or classes
at the designated times. They are also polled for their time and course
preferences but are less likely to receive their first choice than are the
faculty.  Once again, schedule can be used to check for appropriateness of
assignents  and for conflicts, and to produce various written reports 
describing the
state of the schedule.  

\section{Managing the Data}

In this section we describe in detail the content of the data files and the
schedule batch file. Each of these files is a plain ascii file with the 
following common features: Comments may be included by making the first
character of the line the \# character. Long lines may be continued by
making the last character on the line a $\backslash$. For example,
the lines
\smallskip
\begin{verbatim}
These are the times\
that try men's souls. 
\end{verbatim}
\smallskip
would be interpreted by the program as the single line ``These are the
times that try men's souls.'' In general, the content lines of the
data files (i.e., those lines which are not comments) consist of a number
of fields separated by whitespace (spaces and tabs.) If whitespace must
be included within a field then the field must be surrounded with quotes.
For example, if the name of a faculty member is considered to be a single
field, then the name Terry R. McConnell should be entered as ``Terry R. McConnell''. 

The \$ symbol has a special significance. Recall that the configuration
file contained various lines similar to the line ``set MSDOS ON''. The
function of such a line is to define an internal variable (  a 
{\it shell variable}) called, in this case, ``MSDOS'' and
to set the value of this variable. As soon as such a variable is defined,
any subsequent references of the form \$NAME in data files, scripts, or
at the command line, get replaced by the value of the variable called
NAME. Thus, if
the line
\smallskip
\begin{verbatim}
Operating System is $MSDOS 
\end{verbatim}
\smallskip
occurs in a data file, the program actually interprets the line as
``Operating System is ON''. One of the common uses of shell variables is
to define {\it macros} -- short sequences of letters that stand for longer
sequences, thus saving typing. For example, if one defines a shell variable
MGK by set MGK ``Maynard G. Krebs'' then \$MGK can be used subsequently 
instead of typing out Maynard G. Krebs. 

Suppose we wanted to give the program the line ``ONES, TWOS, AND THREES''
by entering ``\$MSDOSES, TWOS, AND THREES'', thinking that the variable
MSDOS in \$MSDOSES would get expanded to ON. Unfortunately, this will
not work because the program will think you are trying to refer to
a variable named MSDOSES. For this reason, brackets are allowed set off the
name of a shell variable embedded in a larger word. If the variable NAME
having value VALUE occurs as \$\{NAME\}foo, then this will be expanded as
VALUEfoo. 

As we have seen so far, the characters \$, \{, \}, and " have special
significance. Another character in this class is the carriage return character
that occurs at the end of each line of text. Suppose you actually wanted
to include one of these characters in a data file. For example, suppose
that you wanted to express an amount of money in a data file (there is
currently no reason to do this ) and entered \$100. The program would try
to expand a variable named ``100'', which is not what you intended. If you
need to use one of the special characters as an ordinary character, you 
must precede it with the $\backslash$ character. Thus to enter the above
symbol for one hundred dollars you would type $\backslash$\$100. Because
of this feature we must now add another character to the list of special
characters: the $\backslash$ character itself. To actually use a 
$\backslash$ literally in a file you would have to enter $\backslash\backslash$.
 Things can get a bit mind-boggling. How do you suppose the program would
interpret $\backslash\backslash\backslash\backslash$\$MSDOS ? (That's
right,  $\backslash\backslash$ON.)

The reader should consult the data files distributed with the program for
examples of usage. The author has made an effort to make these files as
self-documenting as possible by including numerous comment lines.
  
\subsection{The time File}

The main purpose of the time file is to define the possible meeting times
of classes. It also allows the user to specify the start and end dates for
a semester. 

In general, the program recognizes most reasonable formats for dates and times.
 The date 9/12/96 can be entered variously as 9/12/1996, 9-12-96,
9-12-1996, ``9 12 96'', and ``9 12 1996''. After the year 2000, all years
should be given in 4 digit form. 
Times can be entered in civilian
and military format with a variety of variations. For example, the following
are all valid ways to specify 6:00 in the evening: 18:00, 18:00:00, 6:00 pm,
6:00 p.m., 6:00 PM and 6:00 P.M. (In fact, the program only requires that
a p or P follow the 6:00 for a pm time, and that an a or A follow it for
an am time. ) Midnight can be given as 0:00, or 12:00 am. Noon can be
given as 12:00, or 12:00 pm.

Each content line of the time file must begin with one of the following
key words: SemesterStart, EndOfClasses, EndOfSemester, Holidays, 
PeriodicTimeSlot and NonPeriodicTimeSlot. Depending on which keyword 
begins the line, the program expects varying numbers of fields to follow
it. In the following we describe each of these possible lines and their
functions.
\begin{description}
\item [SemesterStart] This should be followed by a date string and a time
string, for example, 
\begin{quote}
SemesterStart 9/2/96 0:00
\end{quote}
says that the semester begins at midnight on the  date. (Let us adopt the
convention that a day begins at midnight.)
 You should
always give the date of Monday of the  week that classes begin and you will probably
want to designate midnight as the time. (See below under PeriodicTimeSlot
to see why Monday at midnight is a good choice.)
\item [EndOfClasses] This should be followed by the date of the last day
of classes and any time of day after the last class will meet on that day.
\item [EndOfSemester] Designate a date and time after the end of all final
examinations.
\item [PeriodicTimeSlot] This is used to specify the set of dates and times
that a given class will meet throughout the semester. Dates and times listed
on this line give the start and end of each class period during the first
week of classes. The program then uses these to generate the set of all
meeting times by assuming that the class will meet the same hours each week.
There are two formats allowed for the rest of a PeriodicTimeSlot line. In
both formats the keyword is followed by a name and a description field. The
name field should be a short, unique, mnemonic for the time slot. It will
be used to refer to that timeslot in other data files. The description is
a string that will be printed in reports. For example, you might name a
timeslot mwf8 and have as its description ``MWF 8:30-9:25'' (Recall that
any field containing space must be surrounded by quotes.) The rest of the
line depends on which format is used.
\begin{enumerate}
\item  The first format is used to describe ``standard'' class meeting times
which will not change from one semester to the next. There should be a list
of integers, two for each class period during the first week of classes. Each
integer represents a number of seconds since the instant specified in
SemesterStart. Thus, the first pair gives the time in seconds from 
SemesterStart until the beginning and end of the first class period. For 
a class that meets five times a week there would be five such pairs. Note
that as long as SemesterStart is given as a Monday at midnight, there should
be no reason to change these lines from one semester to the next. (There is
a list of numbers of seconds since midnight for various ``common'' times in
the file examples/offsets.)
\item The second format is the same, except that the beginning and end of
each class period the first week is given by an explicit date and time. The
advantage of this format is that it is more readable and numbers of seconds
need not be computed. On the other hand, these entries must be updated 
each semester.
\end{enumerate}
\item [Holidays] The function of the holidays line (there should be only one) is to define periods of time during the semester when no classes meet. When
the program creates a periodic time slot (using either of the formats described
above) it  automatically removes any generated class period which overlaps  one of the time periods defined on the Holidays line. Following the
keyword are as many 5-field entries as desired, one for each holiday. In
each case one gives the name of the holiday followed by a date time pair 
giving the time of the beginning of the holiday period and a date time pair
marking the end. For example, suppose classes are dismissed at noon on
the Wednesday before thanksgiving. The thanksgiving entry for 1996 could
then be given as "Thanksgiving 11/27/96 12:00 12/2/96 0:00. As you may recall,
the start of a semester should always be given as a Monday at midnight, but
sometimes classes don't meet the first week of class on Monday. This could
be handled by marking the Monday of the first week of classes as a holiday. 
\item [NonPeriodicTimeSlot] These lines are used to define meeting times
for classes or events that occur only once or a small number of times. After
the keyword one gives a name and description followed by as many 4 field entries as desired. Each such entry
includes a date time pair giving the start of a class time, and a date-time
pair giving the end. For example, the line 
\smallskip
\begin{verbatim}
NonPeriodicTimeSlot fp1 "Final Exam Period 1"  12/20/96 5:00pm 12/20/96 7:00pm
\end{verbatim}
\smallskip
defines the meeting time for a final examination. Time slots created by the
NonPeriodicTimeSlot line are not checked against holidays. This is to allow
you the possibilty of forcing certain events to meet on a holiday.
\end{description}
The new semester program (see the discussion of the Action/New Sem menu choice
below) creates a new time file but does not attempt to update method 2
PeriodicTimeSlot lines or NonPeriodicTimeSlot lines. Instead, they are marked
as comments, with several special comment lines inserted around them to
draw the user's attention.

\subsection{The people file}

The people file is used to give information about all the personnel who will
teach or hold an assignment during the given semester. There is one line
per person and each line has exactly 9 fields. In field 1 give a short name
(20 or fewer characters) for the person. This name must be unique
and is used to refer to that person in other data files. A much longer
(up to 80 character) name can be given in field 9. The longer name appears on
certain reports generated by the program. The remaining fields are
the following:
\begin{description}
\item [Appointment Type] Give a short name for the type of position held by
this person. The program does not expect any particular names -- the choice
of these names is entirely up to the user. However, the names chosen should
correspond with possible entries in the appointment type field of the jobs
file (see below.) When a person is assigned to do a job the program checks
to see whether the person's appointment type is suitable for that job by
comparing the appointment type fields. In any departmental setting there
would almost certainly be at least two types: Faculty and TA, say. However,
these types might be further subdivided to correspond with various levels of
experience.
\item [start] Give a date that represents the time that this person's appointment began. In conjunction with the next field this is intended to allow the
user to indicate that a person might not be available for an entire semester.
If one enters NA for this field then the program will replace it with
SemesterStart. It is good practise, however, to enter actual and realisitic
dates for both this field and the next, since that will allow the program 
to catch certain mistakes. (Do not enter a date before 1-1-70.)
\item [end] See the previous field description. If NA is entered for this
field the program replaces it with EndOfSemester.
\item [seniority] The program does not currently use this field. It may be
safely left as 0.
\item [load fwd] The program does not currently use this field. It may be
safely left as 0.
\item [Semester Load] An integer or decimal value representing the maximum
teaching or assignment load that can be assigned to this person for the
given semester. In most departmental settings there is an agreed-upon level
of work expected from each person, e.g. 2 ``courses'' for a faculty member.
The program does not measure load in any intrinsic unit such as ``courses''
or ``hours.'' Indeed, the number entered here only acquires meaning when
compared with the corresponding entry in the jobs file. At the time
this program was written, our department measured the teaching of a normal
stand-alone class as 6 load units. This was because there were two 
other possible teaching jobs, each involving running a ``recitation'' section,
that were considered to be half as difficult, and one-third as difficult, as
teaching a class. These assignments thus carried loads of 3 and 2 
respectively. The program will complain when any person's total accumulated
load from their assigments in the given semester exceeds the value listed
here, but will make the assignment anyway. It is good practise to start
out with this value set to some departmental minimum, and to raise it to
the actual load assigned only in response to program complaints. That will
prevent you from inadvertantly assigning someone too high a load. 
\item [Academic Year Load] The total load this person may receive in an
academic year. (Fall and Spring semester.) This value should be determined and
entered as soon as possible in the scheduling cycle, and should not be
changed thereafter except to correct errors. 
\end{description}

\subsection{The jobs file} 

This file is used to provide information about the various jobs that can
be assigned to a person in a given semester. Like the people file it has
a very simple structure: there is exactly one line containing 6 fields for
each job. The first field must be a unique short ( less than 20 character )
name for the job. The job will be referred to by this name in other
data files. The 5th field is a more verbose (up to 80 character) description
of the job. The remaining fields are as follows:
\begin{description}
\item [Appointment type] This field describes the type(s) of appointment that
is qualified to do the job, and corresponds with one of the possible 
entries in the field of the same name in the people file. It is often the
case that more than type of appointment is qualified to do a job: for example
a Freshman class might be taught by either a faculty member or a TA. You may
enter multiple appointment types in this field by joining them with the
symbol $|$ with no space between. (The $|$, or vertical bar, symbol is often
used in programming languages to stand for ``or''). 
\item [start] This is a date used to indicate the effective date on which
a job begins. For example, in a team-teaching situation, the second teacher's
job might begin sometime in the middle of the semster. This is, of course, a
rather unusual situation. Most jobs begin at the beginning of the semester and
end at the end of the semester. Recognizing this, one can enter NA in this
field, and the program will replace it with SemesterStart. 
\item [end] The date used to indicate the effective ending date of a job. If
NA is entered the program will replace it with EndOfSemester.
\item [load] An integer or fraction representing the load carried by this job.
See the rather extensive discussion of load in the section on the people file.

\end{description}

\subsection{The rooms file}

This file is used to provide information about the rooms in which classes
can be assigned to meet in the given semester. It has one line containing 6
fields for each room. The first field is a short (less than 20 character)
name for the room. It must be unique and is used to refer to this room in
other data files. The last field is a more verbose (up to 80 character)
field which may be used to give a long name for or to provide a description
of the typical usage of the room. The second and 3rd fields give the name
of the building and the room number. (The number need not actually be
a number -- it is any designation for the room that locates it uniquely in
a building, e.g. B5 for room number 5 in the basement. ) These fields must
be less than 20 characters long. The remaining 2 fields give the seating
capacity of the room for classes and for examinations respectively.

\subsection{The schedule batch file}

This is the file that actually ``creates'' a schedule by specifying what
classes and assignments (e.g. grading jobs ) exist and how they are 
organized into courses. Perhaps a word of explanation as to why this is
called a ``batch'' file is in order. Most users will invoke the various
features of schedule via the menu interface described in the next section, but
it is also possible to run the program in a command line mode. In its command
line mode the program presents the user with a prompt and waits for the
user to type any of a set of possible commands. Some of these commands
cause internal data structures to be allocated which contain information
describing a class, a course, or an assignnment. Thus, such a command can
be said to ``create'' a class. The program can also be directed to read
its commands from a file rather than directly from the user. Such a file
containing a list of commands that might also have been typed at the program's
prompt is called a {\it batch file.} Thus, the schedule batch file is
a file containing the commands necessary to ``create'' a schedule for a
given semester.   

The schedule batch files are located in the data directory. There is one
for each semester and they are named sched.xxx, where xxx is the 3 character
designation for the semester.  After 
reading this section the user should examine one of the existing
batch files as an example illustrating how the commmands are used. 

If a command requires a field for which no value is currently available, 
e.g., the name of a teacher when the teacher has not yet been assigned, the
user should supply ``TBA'' or whatever string has been given as the
value of the shell variable MISSINGDATA. (The default value of MISSINGDATA
when the program starts up is ``TBA'', but the user may change this
at any point in the batch file by using the set command. For example, 
including the line ``set MISSINGDATA FOO'' changes the missing data 
designator from that point forward to ``FOO''.)

Each command in a batch file must occur on a single line, unless that line
is continued using the $\backslash$ character. Any number of blank spaces 
can be included between fields, provided there is at least one blank separating
one field from another. (There can even be any amount of blank space at
the beginning of a line. )

The first content line of any batch file should be a line of the form
\begin{quote}

schedule xxx TITLE
\end{quote}
where xxx is the 3 character designation for the semester and TITLE is a
string which will be printed at the top of the program's listing of the
schedule. (Normally, this will contain spaces and should therefore be
surrounded with quotes.)

The last content line should be a line that reads
\begin{quote}
end ECHO ON
\end{quote}

To understand the remaining lines in a batch file it is necessary to understand
schedule's view of the building blocks of a schedule. These building blocks
are the following:
\begin{description}
\item [Courses] A course is a nonempty collection of classes and assignments.
This organization reflects the fact that real courses often contain multiple
sections which run more or less independently. Each course has a name and
a person who is designated as supervisor of that course.
\item [Classes] A class is a meeting of a group of students with a single
teacher that occurs at specified times in a specified room. A class may have
other classes attached to it, and may also have assignments attached to it.
For example, a lecture class may have a designated recitation section (also
a class) that goes along with it. 
\item [Assignments] An assignment is any job a person can be assigned that
does not involve being the teacher of a class or supervisor of a course. 
\end{description}

Let us now describe each of the commands used to describe the structure of 
a schedule in detail.
\begin{description}
\item[The course command] The course command is introduced by the keyword
{\it course} and is followed by two fields. The first field is used to
supply a name for the course and the second gives the name of a
faculty member who supervises the course. (The name should be one of the
unique identifiers given as the first field in the people file.) The command
also sets the value of an internal variable called CurrentCourse to the name
of the course. This value will remain unchanged until the next course
command. All class and assign commands issued until the next course command
will create classes and assignments which are ``attached'' to the current
course. 

Of course, sometimes one wishes to create ``stand-alone'' classes, or
assignments which service the entire department rather than a particular
course. To handle this situation, a special form of the course command
is provided. The command {\it course end} causes the CurrentCourse value
to be cleared. Any classes or assignments created after such a command, and
before the next course command, will not be attached to any course.

Final examinations may be attached to courses by treating the final examination
as a special class that meets only once. One would need to create a
nonperiodic time slot for the final examination time by creating the appropriate entry in the time file. 

\item[The class command] The class command is introduced by the keyword
{\it class} and is followed by 5 required fields and one optional field.
The required fields, which we shall call name, room, teacher, time, and job,
must be given in that order, and the optional field must follow the 
required ones. The name field is any name the user chooses to give the
class being created. ( For example, the number by which the class is
identified in the time schedule of classes might be used as a name. ) The 
name must be unique within a given course. The room, teacher, time, and
job fields should be chosen from among the unique identifiers which form
the first fields of lines from the rooms, people, time, and jobs files
respectively. (Or the missing data symbol.) They identify the place the
class meets, who teaches it, the times the class meets, and the job entailed
by teaching the class. The optional field may be either the number of students
enrolled in the class (if that information is available) or the symbol ``N''
which indicates that the class is not available for student enrollment. 
(This allows the user to create ``contingency'' classes which are assigned
rooms and times, thus securing those rooms and times against use by other
departments, and which can later be opened to students if the need arises. These classes have no enrollment and are not counted by the program's
auditing routines. ) 

Like the course command, the class command sets the value of an internal
variable called CurrentClass. 

\item[The attach command] The syntax of the attach command is the same as
the class command and it also creates a class attached to the current course.
The only difference is in the way enrollments are handled: any enrollments
entered with the class command are included in the total course enrollment
( or parent class enrollment in the case of a subclass,) whereas enrollments
in an attachment are not included. This is useful, e.g., in a situation
where every student in a course must enroll in a lab in addition to a class.
Defining the lab with the attach command rather than with the class command
keeps its enrollment from being double counted. Common final examinations
in a course should be created with the attach command. Since final 
examinations meet only once it will be necessary to create time slots for 
them in the time file.
\item[The subclass command] 
Sometimes one needs to attach classes or assignments to individual classes
rather than to a course. For example, a course might consist of 3 large lecture
sections ( each of which would be considered a ``class'' ) and the 
lectures might in turn have recitation sections attached to them, or graders
designated to serve that lecture alone. To allow for this kind of structure
the {\it subclass} command is provided. Sublcass causes all subsequent
classes and assignments to be attached to the class which was current at the
time the subclass command was given, up until the command {\it subclass end}
is given. ( Sublcass $\dots$ sublcass end pairs can also be nested up to
five levels deep, although it is unlikely that more than 2 levels would ever
be required.)

\item[The refno command ] This command is used to attach a unique identifying
number to a class ( for example, it can be used to specify the University-wide
unique identifying code for the class. ) The usage is {\it refno string},
which will assign string as the reference number of the current class. Thus,
an appropriate place to include this command is immediately following the
class command that creates a particular class. The reference number, if 
assigned, will appear in the schedule report generated by the program, but
has no other significance.

\item[The assign command]. This command creates an assignment attached to
the current class if it appears between subclass and subclass end. Otherwise
the assignment is attached to the current course if there is one, or else
the assignment created is stand-alone. Following the keyword {\it assign}
there are three fields: name, person, and job. The name is chosen by the
user to identify the assignment. The person and job fields should be chosen
from among the unique identifiers in the people and jobs files. 

\end{description}

Besides the commands described above, any shell command can be included in
the batch file. (See the schedule manual page for documentation on all shell
commands. ) Other than the set command, which can be used to define macros,
it is unlikely that any of these additional commands would be useful. 

It should be mentioned that course, class, schedule, and assign commands
can be used at the shell prompt with only one ( or even with no ) fields
following the command name. These forms of the commands, which are
documented in the schedule manual page, can be used to obtain listings of
various specific parts of the schedule. Because of this, if one includes
by mistake one of these forms in a batch file, then unexpected output
may be produced.  

\section{Using the Program}

Let us assume that the executable schedule.exe has been placed in a directory
that is on the command search path. If that is the case, the program 
can be invoked by merely typing ``schedule'' at the DOS command prompt. 
The program will then present the user with a menu-driven interface that
is described more fully below. There are also various flags that be
given on the command line. The full usage syntax is:
\begin{quote}
schedule [ -c -n  -wd $<$path$>$ -sem $<$string$>$ -debug -help -version ]
\end{quote}
Square brackets indicate that what is inside is optional. Flags that have
an expression in brackets after them must be accompanied by an argument 
of the sort indicated in the brackets. 
The meaning of the flags is as follows:
\begin{description}
\item [-c] Start up the program in command mode. The command mode is 
extensively documented in the Unix-style manual page document called
schedule.1 and will not be discussed further here.
\item [-n] Don't read the file schedule.rc at startup.
\item [-wd] Use the item that follows on the command line (designated as 
$<$path$>$ above ) as the path to the program's working directory. 
\item [-sem ] Use the item that follows as the default semester. (Note that
if either -sem or -wd is used, the settings they cause will override any
settings in schedule.rc.)
\item [-debug] Start the program in debug mode.
\item [-help] Print a usage message and exit.
\item [-version] Print the program version number and exit.
\end{description}

The options c and n can be given simultaneously as either -nc or -cn. 

\subsection{The menu interface}

The interface conforms to what is commonly called the {\it document-view}
model. At any given moment there is always a document on view in the main
window. Various actions, such as saving the document, printing it, and
switching to a different document, can be selected from the menu and its
submenus. One can toggle back and forth between the document and the menus
by using the $<$esc$>$ key. There is a status bar at the bottom of the screen
that displays important aspects of the program's current state. The title
line at the top of the screen is always visible and shows the program
version number. 

When the cursor appears in the main window, the arrow keys can be used to
move around within the document on view. In addition, the following keys
may  be used: Page Down, Page Up, Control-End (takes the cursor to
the end of the document,) Control-Home (takes the cursor to the beginning of
of the document,) Enter (scrolls the screen one line down), and space-bar
(same as page down, except that the bottom line of the last screen is
retained as the top line of the next.) As mentioned in the last paragraph,
the escape key will activate the main menu which appears on the line
above the main document window. 

Once the menu is activated, the left and right arrow keys may be used to 
highlight various menu items. The only other keys which are valid in this mode are the escape key and the down-arrow key. The escape key returns the user
to the document window, and the keys then behave as described in the previous
paragraph. The down arrow key causes a submenu to pop down from whatever
menu item is currently highlighted and places the program in submenu mode.

In submenu mode the up and down arrow keys cycle the user among the various
choices available on that submenu. The only other keys active are the
escape key and the enter key. The escape key takes the user back to the
main menu mode described in the previous paragraph. The enter key selects
the currently highlighted submenu item. 

In each of the three modes -- main window, menu, and submenu -- invalid
keystrokes will cause a beep, but will otherwise do no harm. 

Several of the menu choices cause a special dialogue window to appear at the
right center of the screen. Each such window has a descriptive title at the
top, a prompt line at which the user can enter a response, and one or two
lines of information below that explain what type of response is expected
from the user. The user can edit their response only by using the backspace
key. Once the response is typed, striking the enter key will submit it to
the program for processing. If instead, the user strikes the escape key 
whenever a dialogue window is displayed, the dialogue will be canceled and
the user returned to the main menu level. Any error messages to the user
during a dialogue will appear on the status bar and will remain until the
next keystroke. 

The status bar displays the following information, from left to right: 
First is 
the current position, L:C, in the file, where L is the line number and C is
the column number. Next is the name of the currently open file, i.e, the
one on view in the main window. Normally this name will not be meaningful
to the user since the file being viewed is a temporary file created by
the program. Other files can be opened from the file menu and from the help
menu.
 The current semester is displayed next. For example if one sees SEM: f97
on the status bar, the next time data is loaded all data will be taken
from the data files for the f97 semester. Note, however, that the 
presence of the SEM value on the status bar does not indicate that any data
has been loaded. If the data from some semester has actually been loaded
then that will be indicated on the status bar by [sem]. For
example, SEM: f97 [s96] indicates that the Spring 96 data is currently
loaded, but that the current semester is f97. (The main significance of the
current semester is in how it influences the behaviour of several menu choices --
see below. ) Next is the current working directory of the program. Finally,
an additional message may be displayed, the default being a message that
reminds the user of the significance of the escape key.

Let us turn now to a description of each of the possible menu choices.
(Note that in some cases menu items are marked as xxxxxx. These are menu
slots which are not currently used by the program. Choosing them is
harmless, but unproductive.)

\begin{description}
\item[The file menu] Most of the choices on this menu have to do with
manipulation of the currently open file (the one on display in the main
window and named on the status bar.)    

\begin{description}
\item[Open] This choice begins a dialogue which asks the user to name an existing
file which will then be opened and displayed in the main window. The user
can give either a path relative to the current working directory of the
program, or an absolute path beginning with $\backslash$. The chosen file
becomes the current open file until supplanted by another open or other menu
choice that replaces the current open file. 
\item[Load] This choice causes the data for the current semester ( the one
displayed after the SEM symbol on the status bar ) to be loaded into memory.
It also causes the batch file for the current semester to be processed. 
Any error messages generated during the data load or the batch file processing
are placed in the program's temporary file and the temporary file is made
the current open file so that they may be viewed at leasure.
\item[Save As] This command is used to save the current open file to the
computer's hard drive or floppy drive. The current open file is often a
temporary file which will be removed when the schedule program quits, so
it is necessary to choose this item if the user wants to keep copies of any
output that schedule produces. Choosing this item will start a dialogue in
which the user is asked to provide a name (actually, a path) to save the
current open file under. If the user provides the name of a file that
already exists he will be asked to confirm whether he wishes to replace
the existing file and whether he wishes to replace the file or add the
current open file to the end of it (append.)
The name given is taken as a path relative to the current
working directory, or as an absolute path if it begins with the $\backslash$
character.
\item[Print] This command causes the current open file to be printed on
the system printer (more precisely, the text is directed to the PRN device.)
Before the file is printed, however, the user is asked in a dialogue whether
to print in landscape mode or in normal mode. Normal mode is a reasonable
choice for all files generated by the program, with the exception of a
roomchart file. Those work best when printed in landscape mode with a
compressed font. In order for all of this to work, the program needs to
be told the printer commands necessary to put the printer in landscape mode
and to return the printer to portrait mode. This can be done by including
the lines ``set LANDSCAPE xxxxx'' and ``set PRINTERNORMAL xxxxx'' in the
schedule.rc file. Here xxxxx should replaced by whatever escape sequence is
necessary for the printer attached to the computer (the user should consult
her printer manual for this information.) The schedule.rc that comes with
the program already contains settings that should work with most Hewlet-Packard
printers. 
\item[Quit] This command really needs no explanation. From time to time,
schedule may shut down abnormally (e.g., if there is a power failure.) If this
happens then the program will not have time to remove its temporary file. If
you find a file with a strange name (e.g., TEMP.\$\$\$) in the working
directory, this is the likely explanation. Simply remove this file when
schedule is not running.
\end{description}
\item[The View menu]
The commands on this menu are primarily used to change the file on view in
the main window. They are the means by which schedule produces reports 
summarizing various aspects of the data.
\begin{description}
\item[Schedule] This choice causes schedule to generate a report listing all
courses, classes, and assignments of the current semester. It will only work
if the data for the current semester has been loaded. If the user wishes to
retain a copy of this document she must either print it or save it in a file.
Both of these can be done from the File menu.
\item[Audit] This choice causes schedule to generate a report comparing 
demand for the various possible jobs with the availability of each job as
measured by load units. The figures are displayed for each job and for
each semester of the current academic year. In addition, a final column 
totals the figures for each of the two semesters. This menu choice will
work regardless of whether any data has been loaded or not.

 Before the report is generated, the user is asked
in a dialogue to specify how each person's load for a semester
should be determined. In one method, the program uses one half the academic
year load in each of the two semesters, and in the other the load for each
semester, as read from that semester's people file, is used. Which method is
the appropriate one to use depends on the status of the scheduling process
at the time the program is run. Normally, the semester load figure will not
be accurate, especially for the Spring semester, until all job assignments
for the Fall Semester have been made and the Spring semester's load
figures  updated to reflect the load left over from what was
assigned in the fall. On the other hand, one often wants to get a reading
on whether the department is on target for meeting its teaching committments
well before all Fall assignments have been made. In this situation one should
choose the average of academic year load method, with the understanding that
only the totals column will be meaningful. In all other situations, one should
choose the actual semester load method. 

The report produced is in 4 sections. The first section, the demand section,
counts how many load units have been allocated to each of the possible
jobs because of commands in one of the schedule batch files. For example,
if one of the rows is titled ``Teacher'' then the entry for the fall semester
column counts how many times classes with the job field ``Teacher'' have
been created in the fall semester batch file, and multiplies that figure by
the load units for the Teacher job as given in the jobs file for that 
semester. It is possible to have different load values for a given job
in each of the semesters, although that is an unlikely situation. At the
end of the demand section there is a total line which gives the total of
the demands in the column above it.

The second section, the availabilty section, counts how many load units
are available for each job. For example, if one of the rows is titled 
``Teacher'', then the entry for the Fall semester counts up the load units
in the given semester ( as computed by the method chosen by the user )
for all people whose appointment type allows them to do the given job.
In this section, the totals line is {\bf not} the total of the entries above,
but rather gives the total of all load units of all people in the people file
for the given semester. 

The third section shows the difference between each of the lines in the
demand section and the corresponding line in the availability section.

Finally, the legend section displays the number of load units that each of
the various jobs is worth.  
\item[RmChart] This causes a roomchart to be created for a date in the
current semester. Like the audit report, no data needs to be currently loaded
in order for this to work. The user is asked in a dialogue to supply a date
and time for the beginning of the time period covered by the room chart, and
a date and time for the end of the time period. (Normally, both dates would
be the same.) For example, suppose there is an existing batch file sched.f97
and one wishes to create room charts showing how rooms are being used
on Monday-Wednesday-Friday  and on Tuesday-Thursday (assuming, as is typical,
that room usage is pretty much the same on Monday, Wednesday, or Friday and
on Tuesday or Thursday.) One would begin by choosing any date during the
f97 semester that is on a Monday, Wednesday, or a Friday, and on which classes
actually meet. For example, one could choose September 11, a Wednesday.   
One then selects a range of times wide enough to include all classes that
meet on that day. Thus, suppose no classes meet before 8:00 am or after 
7:00 pm. One then could specify 9/11/97 and 8:00 as the start of the time
block, and 9/11/97, 19:00 ( or 7:00 pm) as the end. After saving or printing
the resulting room chart, one would repeat the entire procedure for a Tuesday
or Thursday date. 

The roomchart that will appear in the main window will not be intelligible 
because the long lines in the chart will be wrapped. To view the chart it
is necessary either to print it in Landscape mode or to save it in a file
and view it with an editor that supports horizontal scrolling (e.g. DOS edit.)

The format of the information displayed in a roomchart should be self-explanatory. 

\item[People] This command creates blocks of information, one for each person
listed in the people file of the current semester. The data for the current
semester must be loaded in order for this command to work.
A numbered list of all
such people is first displayed in the main window, and a dialogue which allows
the user to select a person or persons from that list is started. One can
either choose to display information about all people by entering the word
all at the dialogue prompt, or to display information about a person or
a range of people by entering that person's number, or a range of numbers,
at the prompt. What if you want to display information about Mr foo, but
you don't remember Mr foo's number and his name is not visible on the first
page ? In that case, escape from the dialogue by hitting the escape key, and
then scroll down to Mr foo's entry by using the arrow keys. Finally, start
up the dialogue again by choosing the People menu item a second time. (Another
idea would be to print out a list of the people for future reference by 
escaping from the dialogue and printing the numbered list of people from the
File menu. Remember, anything that is visible in the main window can be
printed or saved!) 

Each block of information created after the dialogue is finished concerns 
a particular person. Listed are the appointment dates, the load values, and
the jobs assigned to that person in the given semester. The blocking of
information is done so that a printout of this report can be cut up and
distributed to each person to inform him of his status in that semester.
\end{description}
\item[The Set menu] The choices on  this menu allow the user to change
the values of certain internal variables that influence the behavior of the
program.
\begin{description}
\item[Semester] This selection starts a dialogue that allows the user to change
the current semester. Recall that the value of the current semester is
displayed at all times on the status bar, and determines which data is loaded
when the user selects load on the file menu. The value of the current
semester when the program starts up can be controlled by including the
line ``set SEM xxx'' in schedule.rc, where xxx is any desired 3 character
semester designator. 
\item[Vars] This selection displays the names and values of all currently
defined shell variables in the main window, and starts a dialogue which allows
the user to change the value of one of the existing variables, or to define
a new one. Note that SEM, WD and MISSINGDATA are shell variables, so anything
that can be set by using any of the choices on the SET menu can also be set
using the Vars menu choice. 
\item[Dir] This selection allows the user to change the working directory of
the program. The value of the working directory is displayed at all times
on the status bar. Recall that whenever the user is asked to supply a path
to a file, the path will be taken relative to the working directory unless
it begins with the $\backslash$ character. The user should not change the
working directory without good reason, and should always change back to
the ``official'' working directory as soon as possible. Many features of the
program will not work unless the working directory is chosen correctly.
\item[TBA] This selection allows the user to change the value of the
internal variable called MISSINGDATA. The default value of this variable,
and the value it will have at startup (unless changed in schedule.rc) is 
``TBA''. The significance of the MISSINGDATA variable is twofold. It can
be used anywhere in a schedule batch file that data is called for when the
data is not available. For example, with the class command one must supply
at least 4 pieces of data: the room, teacher, time, and job. If any of these
values is unknown or undetermined, the missing data symbol can be entered
instead. Secondly, the missing data symbol will appear in reports wherever
there there is unknown or missing data. Note that, because of this twofold
significance, it may be useful to change the missing data while the program
is running. For example, one symbol may have been used in the schedule batch
file, but one might prefer a different symbol to appear on reports. To do
this, simply use this menu choice to change the missing data symbol after data
has been loaded, but before a report is generated.  
\end{description}
\item[The Action menu] This menu allows the user to escape from the user
interface in order to access other facilities of the schedule program. 
\begin{description}
\item[Shell] This selection allows the user to escape from the menu interface
to the command line interface. The command line interface can also be
reached by starting the program with the -c flag. The command line interface
is very powerful, but the casual user will probably never need to use it.
See the Unix-style schedule manual page for extensive documentation on the 
command line interface. To exit from the command line interface and return
to the menu interface, enter ``end'' at the command line prompt. To exit
from the command line interface directly to the DOS prompt, enter ``exit'' at
the prompt. 
\item[New Sem] This selection runs an external program called newsmstr.exe
which is used to begin the process of setting up the data files and the batch
file for a new semester. This program can also be run on its own, independently
of the schedule program: go to the src/newsmstr subdirectory of the
working directory and type ``newsmstr''. The newsmstr program is very verbose,
and leads the user step by step through the process of creating new data
files. For further information, consult the newsmstr manual page. 
\end{description}
\item[Help] This menu provides access to online documentation. It is very
incomplete at the present time -- the only working selection displays a file
called info.txt in the main window which contains some general information
about the program.
\end{description} 

The first time a new semester's data and batch file is loaded it is normal
to have many error messages displayed in the main window. An efficient way
to correct the errors is to have the schedule program running in one DOS 
window ( say, under MS -Windows 3.1 or 95 ) and to have an editor (e.g,
MS-Word, or DOS Edit ) running in another. Correct any problems in the
data and batch files in the editor window, save the changes, and the choose
load again from the menu in the schedule program window. After several
iterations of this process, the data should load cleanly. Remember that if
you use a word processor for editing it is necessary to save the data
and batch files as plain ascii files. 
 
\section{Official Bugs}
There are two kinds of bugs: known bugs and unknown ones. This program, of
course, has both kinds. For a listing of the known bugs consult the
schedule manual page. If you find any of the latter kind, please report
them to the author at the address listed in the info.txt file.

\end{document}
